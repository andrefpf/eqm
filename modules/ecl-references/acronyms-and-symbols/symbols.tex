%!TEX root = ../main.tex

% @Author: ismaelseidel
% @Date:   2016-10-21 16:40:29
% @Last Modified by:   Ismael Seidel
% @Last Modified time: 2025-05-27 20:59:35

%% List of Symbols


\newglossaryentry{lengthtree}%
{%
	name={\ensuremath{L_t}},
	description={description here},
	sort={L},
	type=symbols
}


\newglossaryentry{stddev}%
{%
	name={\ensuremath{\sigma}},
	description={Standard deviation},
	sort={sigma},
	type=symbols
}


\newglossaryentry{quantization}%
{%
	name={\matrixname{Q}},
	description={Quantization},
	sort={Q},
%	see=[Acronyms:]{qp},
	type=symbols
}

\newglossaryentry{quantizationParameter}%
{%
	name={\scalarname{qp}},
	description={Quantization Parameter Value},
	sort={QP},
%	see=[Acronyms:]{qp},
	type=symbols
}

\newglossaryentry{transform}%
{%
	name={\matrixname{T}},
	description={Transform},
	sort={T},
	type=symbols
}


\newglossaryentry{inversequantization}%
{%
	name={\matrixname{Q}\ensuremath{^{-1}}},
	description={Inverse Quantization},
	sort={Qinv},
%	see=[Acronyms:]{qp},
	type=symbols
}

\newglossaryentry{inversetransform}%
{%
	name={\matrixname{T}\ensuremath{^{-1}}},
	description={Inverse Transform},
	sort={Tinv},
	type=symbols
}

% orif
% recf

\newglossaryentry{orif}%
{%
	name={\ensuremath{\namedmatrix{F}{ori}}},
	description={Original Frame},
	sort={Frame original},
	type=symbols
}

\newglossaryentry{recf}%
{%
	name={\ensuremath{\namedmatrix{F}{rec}}},
	description={Reconstructed Frame},
	sort={Frame reconstructed},
	type=symbols
}

\newglossaryentry{canf}%
{%
	name={\ensuremath{\namedmatrix{F}{can}}},
	description={Candidate Frame},
	sort={Frame candidate},
	type=symbols
}


\newglossaryentry{reff}%
{%
	name={\ensuremath{\namedmatrix{F}{ref}}},
	description={Reference Frame},
	sort={Frame reference},
	type=symbols
}

\newglossaryentry{sets}%
{%
	name={\ensuremath{S}},
	description={Set of candidate blocks},
	sort={set of candidate blocks},
	type=symbols
}

\newglossaryentry{colordepth}%
{%
	name={\scalarname{b}},
	description={Color depth (number of bits)},
	sort={Color depth},
	type=symbols
}

\newglossaryentry{jcost}%
{%
	name={\ensuremath{j_{\text{cost}}}},
	description={The lagrangian rate-distortion cost of selecting a given candidate as reference},
	sort={J},
	type=symbols
}

\newglossaryentry{jlambda}%
{%
	name={\(\lambda\)},
	description={The Lagrange multiplier.},
	sort={lambdaj},
	type=symbols
}

\newglossaryentry{lambdamode}%
{%
	name={\ensuremath{\lambda^\text{mode}}},
	description={The Lagrange multiplier used during mode decision.},
	sort={lambdamode},
	type=symbols
}

\newglossaryentry{lambdamotion}%
{%
	name={\ensuremath{\lambda^\text{motion}}},
	description={The Lagrange multiplier used during motion estimation.},
	sort={lambdamotion},
	type=symbols
}

\newglossaryentry{lambdapred}%
{%
	name={\ensuremath{\lambda^\text{pred}}},
	description={The Lagrange multiplier used during prediction.},
	sort={lambdapred},
	type=symbols
}

\newglossaryentry{cfactor}%
{%
	name={\ensuremath{c}},
	description={A scaling factor used in SATD. In HM, it is defined as $\sfrac{1}{2^n}$.},
	sort={cfactor},
	type=symbols
}

\newglossaryentry{glsotimes}%
{%
	name={\ensuremath{\otimes}},
	description={The Kronecker product.},
	sort={kronecker},
	type=symbols
}

\newglossaryentry{block}%
{%
	name={\ensuremath{\textbf{B}}},
	description={A given block, i.e., a partition of the frame},
	sort={block},
	type=symbols
}

\newglossaryentry{oriblock}%
{%
	name={\ensuremath{\textbf{B}^{\text{ori}}}},
	description={Original block},
	sort={block original},
	type=symbols
}

\newglossaryentry{canblock}%
{%
	name={\ensuremath{\textbf{B}^{\text{can}}}},
	description={Candidate block},
	sort={block candidate},
	type=symbols
}


\newglossaryentry{recblock}%
{%
	name={\ensuremath{\textbf{B}^{\text{rec}}}},
	description={Reconstructed block},
	sort={block reconstructed},
	type=symbols
}


\newglossaryentry{resblock}%
{%
	name={\ensuremath{\textbf{B}^{\text{res}}}},
	description={Residue block},
	sort={block residue},
	type=symbols
}

\newglossaryentry{refblock}%
{%
	name={\ensuremath{\textbf{B}^{\text{ref}}}},
	description={Reference block},
	sort={block reference},
	type=symbols
}

% differencesM
% difference
\newglossaryentry{errormatrix}%
{%
	name={\ensuremath{\textbf{E}}},
	description={Error matrix. Such matrix can represent the difference between original and decoded blocks or even frames.},
	sort={error matrix},
	type=symbols
}


\newglossaryentry{differencesM}%
{%
	name={\ensuremath{\textbf{D}}},
	description={Differences matrix; \textbf{D} = \canb - \orib },
	sort={differences Matrix},
	type=symbols
}

\newglossaryentry{difference}%
{%
	name={\ensuremath{d}},
	description={a scalar value from the differences matrix; },
	sort={differences value},
	type=symbols
}

\newglossaryentry{transformeddifferencesM}%
{%
	name={\ensuremath{\operatorname{TD}}},
	description={Transformed Differences matrix; },
	sort={transformed differences Matrix},
	type=symbols
}

\newglossaryentry{transformeddifference}%
{%
	name={\ensuremath{td}},
	description={a scalar value from the transformed differences matrix; },
	sort={transformed differences value},
	type=symbols
}



% numbers 

\newglossaryentry{naturalnumbers}%
{%
	name={\ensuremath{\mathbb{N}}},
	description={The set of natural numbers. },
	sort={Number Natural},
	type=symbols
}

\newglossaryentry{rationalnumbers}%
{%
	name={\ensuremath{\mathbb{Q}}},
	description={The set of rational numbers. },
	sort={Number Rational},
	type=symbols
}

\newglossaryentry{integernumbers}%
{%
	name={\ensuremath{\mathbb{Z}}},
	description={The set of integer numbers. },
	sort={Number Integer},
	type=symbols
}

\newglossaryentry{realnumbers}%
{%
	name={\ensuremath{\mathbb{R}}},
	description={The set of real numbers. },
	sort={Number Real},
	type=symbols
}

% \newcommand{\naturals}{\ensuremath{\mathbb{N}}}
% \newcommand{\integers}{\ensuremath{\mathbb{Z}}}
% \newcommand{\reals}{\ensuremath{\mathbb{R}}}

% \newcommand{\primes}{\ensuremath{\mathbb{P}}}
% \newcommand{\rationals}{\ensuremath{\mathbb{Q}}}
% \newcommand{\complex}{\ensuremath{\mathbb{C}}}
% \newcommand{\quaternions}{\ensuremath{\mathbb{H}}}