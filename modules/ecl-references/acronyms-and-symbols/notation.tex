%!TEX root = ../main.tex

% @Author: Ismael Seidel
% @Date:   2019-03-18 11:24:11
% @Last Modified by:   Ismael Seidel
% @Last Modified time: 2019-12-06

\renewcommand*\glspostdescription{}

\newcommand{\newNotation}[3]{%
	\newglossaryentry{#1}{%
		type=notation,
		name={#2},
		description={#3},
		sort={#1}
	}
	% \newacronym[type=important]{#1}{#2}{#3}
}%

\newcommand{\newSubNotation}[4]{%
	\newglossaryentry{#1}{%
		type=notation,
		parent={#2},
		name={#3},
		description={#4},
		sort={#1}
	}
	% \newacronym[type=important]{#1}{#2}{#3}
}%




\newNotation{set}{\setname{A}}{Upper case italic character is a set}

\newNotation{ascalar}{\scalarname{A}}{Lower case italic character is a scalar}
\newSubNotation{scalarfunction}{ascalar}{\scalarfunction{f}{...}}{is a function that returns an scalar}

\newNotation{matrix}{\matrixname{A}}{Upper case bold character is a matrix}
\newSubNotation{namedmatrix}{matrix}{\namedmatrix{A}{name}}{is a named matrix}
\newSubNotation{sizedmatrix}{matrix}{\matrixname{A}$_{m\times n}$}{is an $m\times n$ sized matrix}
\newSubNotation{sizedmatrixm}{matrix}{\matrixname{A}$_{m}$}{is an $m\times m$ sized matrix}
\newSubNotation{indexmatrix}{matrix}{\matrixname{A}$_{i,j}$}{is an element in the $i^\text{th}$ row and $j^\text{th}$ column}
\newSubNotation{sizedmatrixpart}{matrix}{\matrixname{A}$_{m\times n;m'\times n'}$}{is an $m\times n$ sized matrix with $p\times q$ partitions where $p=m/m'$ and $q=n/n'$.}
\newSubNotation{sizedmatrixpartm}{matrix}{\matrixname{A}$_{m;m'}$}{is an $m\times m$ sized matrix with $p\times p$ partitions where $p=m/m'$. 
For instance, an $8\times8$ matrix, partitioned in $2\times2$ submatrices, each with size $4\times4$: \begin{small}
  \begin{equation} \nonumber
  	\setlength{\arraycolsep}{2.5pt}
  	\matrixname{A}_{8;4} 
  	{=} 
  	\begin{bmatrix} [ccc|ccc] 
        \matrixname{A}_{1,1} & {\ldots} & \matrixname{A}_{1,4} & \matrixname{A}_{1,5} & \ldots & \matrixname{A}_{1,8} \\
        \vdots & \ddots & \vdots & \vdots  & \ddots & \vdots \\
        \matrixname{A}_{4,1} & \ldots & \matrixname{A}_{4,4} & \matrixname{A}_{4,5}  & \ldots & \matrixname{A}_{4,8} \\ \hline
        \matrixname{A}_{5,1} & \ldots & \matrixname{A}_{5,4} & \matrixname{A}_{5,5}  & \ldots & \matrixname{A}_{5,8} \\
        \vdots & \ddots & \vdots & \vdots  & \ddots & \vdots \\ 
        \matrixname{A}_{8,1} & \ldots & \matrixname{A}_{8,4} & \matrixname{A}_{8,5}  & \ldots & \matrixname{A}_{8,8} \\
      \end{bmatrix} 
     {=} 
      \begin{bmatrix}[c|c]
      	\matrixname{A}_{;1,1} & \matrixname{A}_{;1,2} \\ \hline
      	\matrixname{A}_{;2,1} & \matrixname{A}_{;2,2}
      \end{bmatrix}
  \end{equation}
\end{small}
}
\newSubNotation{matrixfunction}{matrix}{\matrixfunction{F}{...}}{is a function that returns a matrix}



% \newglossaryentry{vector}{%
%   type=notation,
%   name={a$\overrightarrow{v}$},
%   description={Lower case character with an arrow above is a vector},
%   sort={vector}
% }

\newNotation{vector}{${v}$}{Lower case character with an arrow above is a vector}

\newNotation{zrtlsignal}{\signalname{A}}{Lower case character with typewriter font represents an RTL signal}
\newNotation{zrtlstate}{\statename{A}}{Upper case character with typewriter font represents an RTL state}



\glsaddall[types={notation}]